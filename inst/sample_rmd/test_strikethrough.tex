% Options for packages loaded elsewhere
\PassOptionsToPackage{unicode}{hyperref}
\PassOptionsToPackage{hyphens}{url}
%
\documentclass[
]{article}
\usepackage{lmodern}
\usepackage{amssymb,amsmath}
\usepackage{ifxetex,ifluatex}
\ifnum 0\ifxetex 1\fi\ifluatex 1\fi=0 % if pdftex
  \usepackage[T1]{fontenc}
  \usepackage[utf8]{inputenc}
  \usepackage{textcomp} % provide euro and other symbols
\else % if luatex or xetex
  \usepackage{unicode-math}
  \defaultfontfeatures{Scale=MatchLowercase}
  \defaultfontfeatures[\rmfamily]{Ligatures=TeX,Scale=1}
\fi
% Use upquote if available, for straight quotes in verbatim environments
\IfFileExists{upquote.sty}{\usepackage{upquote}}{}
\IfFileExists{microtype.sty}{% use microtype if available
  \usepackage[]{microtype}
  \UseMicrotypeSet[protrusion]{basicmath} % disable protrusion for tt fonts
}{}
\makeatletter
\@ifundefined{KOMAClassName}{% if non-KOMA class
  \IfFileExists{parskip.sty}{%
    \usepackage{parskip}
  }{% else
    \setlength{\parindent}{0pt}
    \setlength{\parskip}{6pt plus 2pt minus 1pt}}
}{% if KOMA class
  \KOMAoptions{parskip=half}}
\makeatother
\usepackage{xcolor}
\IfFileExists{xurl.sty}{\usepackage{xurl}}{} % add URL line breaks if available
\IfFileExists{bookmark.sty}{\usepackage{bookmark}}{\usepackage{hyperref}}
\hypersetup{
  pdftitle={Colortable},
  pdfauthor={Ellis Hughes},
  hidelinks,
  pdfcreator={LaTeX via pandoc}}
\urlstyle{same} % disable monospaced font for URLs
\usepackage[margin=1in]{geometry}
\usepackage{color}
\usepackage{fancyvrb}
\newcommand{\VerbBar}{|}
\newcommand{\VERB}{\Verb[commandchars=\\\{\}]}
\DefineVerbatimEnvironment{Highlighting}{Verbatim}{commandchars=\\\{\}}
% Add ',fontsize=\small' for more characters per line
\usepackage{framed}
\definecolor{shadecolor}{RGB}{248,248,248}
\newenvironment{Shaded}{\begin{snugshade}}{\end{snugshade}}
\newcommand{\AlertTok}[1]{\textcolor[rgb]{0.94,0.16,0.16}{#1}}
\newcommand{\AnnotationTok}[1]{\textcolor[rgb]{0.56,0.35,0.01}{\textbf{\textit{#1}}}}
\newcommand{\AttributeTok}[1]{\textcolor[rgb]{0.77,0.63,0.00}{#1}}
\newcommand{\BaseNTok}[1]{\textcolor[rgb]{0.00,0.00,0.81}{#1}}
\newcommand{\BuiltInTok}[1]{#1}
\newcommand{\CharTok}[1]{\textcolor[rgb]{0.31,0.60,0.02}{#1}}
\newcommand{\CommentTok}[1]{\textcolor[rgb]{0.56,0.35,0.01}{\textit{#1}}}
\newcommand{\CommentVarTok}[1]{\textcolor[rgb]{0.56,0.35,0.01}{\textbf{\textit{#1}}}}
\newcommand{\ConstantTok}[1]{\textcolor[rgb]{0.00,0.00,0.00}{#1}}
\newcommand{\ControlFlowTok}[1]{\textcolor[rgb]{0.13,0.29,0.53}{\textbf{#1}}}
\newcommand{\DataTypeTok}[1]{\textcolor[rgb]{0.13,0.29,0.53}{#1}}
\newcommand{\DecValTok}[1]{\textcolor[rgb]{0.00,0.00,0.81}{#1}}
\newcommand{\DocumentationTok}[1]{\textcolor[rgb]{0.56,0.35,0.01}{\textbf{\textit{#1}}}}
\newcommand{\ErrorTok}[1]{\textcolor[rgb]{0.64,0.00,0.00}{\textbf{#1}}}
\newcommand{\ExtensionTok}[1]{#1}
\newcommand{\FloatTok}[1]{\textcolor[rgb]{0.00,0.00,0.81}{#1}}
\newcommand{\FunctionTok}[1]{\textcolor[rgb]{0.00,0.00,0.00}{#1}}
\newcommand{\ImportTok}[1]{#1}
\newcommand{\InformationTok}[1]{\textcolor[rgb]{0.56,0.35,0.01}{\textbf{\textit{#1}}}}
\newcommand{\KeywordTok}[1]{\textcolor[rgb]{0.13,0.29,0.53}{\textbf{#1}}}
\newcommand{\NormalTok}[1]{#1}
\newcommand{\OperatorTok}[1]{\textcolor[rgb]{0.81,0.36,0.00}{\textbf{#1}}}
\newcommand{\OtherTok}[1]{\textcolor[rgb]{0.56,0.35,0.01}{#1}}
\newcommand{\PreprocessorTok}[1]{\textcolor[rgb]{0.56,0.35,0.01}{\textit{#1}}}
\newcommand{\RegionMarkerTok}[1]{#1}
\newcommand{\SpecialCharTok}[1]{\textcolor[rgb]{0.00,0.00,0.00}{#1}}
\newcommand{\SpecialStringTok}[1]{\textcolor[rgb]{0.31,0.60,0.02}{#1}}
\newcommand{\StringTok}[1]{\textcolor[rgb]{0.31,0.60,0.02}{#1}}
\newcommand{\VariableTok}[1]{\textcolor[rgb]{0.00,0.00,0.00}{#1}}
\newcommand{\VerbatimStringTok}[1]{\textcolor[rgb]{0.31,0.60,0.02}{#1}}
\newcommand{\WarningTok}[1]{\textcolor[rgb]{0.56,0.35,0.01}{\textbf{\textit{#1}}}}
\usepackage{longtable,booktabs}
% Correct order of tables after \paragraph or \subparagraph
\usepackage{etoolbox}
\makeatletter
\patchcmd\longtable{\par}{\if@noskipsec\mbox{}\fi\par}{}{}
\makeatother
% Allow footnotes in longtable head/foot
\IfFileExists{footnotehyper.sty}{\usepackage{footnotehyper}}{\usepackage{footnote}}
\makesavenoteenv{longtable}
\usepackage{graphicx,grffile}
\makeatletter
\def\maxwidth{\ifdim\Gin@nat@width>\linewidth\linewidth\else\Gin@nat@width\fi}
\def\maxheight{\ifdim\Gin@nat@height>\textheight\textheight\else\Gin@nat@height\fi}
\makeatother
% Scale images if necessary, so that they will not overflow the page
% margins by default, and it is still possible to overwrite the defaults
% using explicit options in \includegraphics[width, height, ...]{}
\setkeys{Gin}{width=\maxwidth,height=\maxheight,keepaspectratio}
% Set default figure placement to htbp
\makeatletter
\def\fps@figure{htbp}
\makeatother
\setlength{\emergencystretch}{3em} % prevent overfull lines
\providecommand{\tightlist}{%
  \setlength{\itemsep}{0pt}\setlength{\parskip}{0pt}}
\setcounter{secnumdepth}{-\maxdimen} % remove section numbering
\usepackage{ulem}

\title{Colortable}
\author{Ellis Hughes}
\date{2/16/2020}

\begin{document}
\maketitle

\hypertarget{colortable}{%
\subsection{\{\{colortable\}\}}\label{colortable}}

Examples of \{\{colortable\}\} in action!

\begin{Shaded}
\begin{Highlighting}[]
\KeywordTok{library}\NormalTok{(colortable)}
\KeywordTok{library}\NormalTok{(tidyverse)}
\KeywordTok{library}\NormalTok{(knitr)}
\end{Highlighting}
\end{Shaded}

\begin{Shaded}
\begin{Highlighting}[]
\NormalTok{cell_sample  <-}\StringTok{ }\KeywordTok{color_vctr}\NormalTok{(}\DecValTok{24}\NormalTok{, }\DataTypeTok{text_color =} \StringTok{"red"}\NormalTok{, }\DataTypeTok{background =} \StringTok{"blue"}\NormalTok{)}
\NormalTok{cell_sample2 <-}\StringTok{ }\KeywordTok{color_vctr}\NormalTok{(}\DecValTok{42}\NormalTok{, }\DataTypeTok{background =} \StringTok{"yellow"}\NormalTok{)}
\NormalTok{cell_sample3 <-}\StringTok{ }\KeywordTok{color_vctr}\NormalTok{(}\DecValTok{68}\NormalTok{, }\DataTypeTok{text_color =} \StringTok{"magenta"}\NormalTok{, }\DataTypeTok{style =} \StringTok{"strikethrough"}\NormalTok{)}
\NormalTok{cell_sample4 <-}\StringTok{ }\KeywordTok{color_vctr}\NormalTok{(}\DecValTok{70}\NormalTok{, }\DataTypeTok{text_color =} \StringTok{"green"}\NormalTok{)}

\NormalTok{cell_sample}
\end{Highlighting}
\end{Shaded}

\texttt{\#\#\ [1]\ \colorbox[rgb]{0.0,0.0,1.0}{\textcolor[rgb]{1.0,0.0,0.0}{24}}}\newline

\begin{Shaded}
\begin{Highlighting}[]
\NormalTok{cell_sample2}
\end{Highlighting}
\end{Shaded}

\texttt{\#\#\ [1]\ \colorbox[rgb]{1.0,1.0,0.0}{42}}\newline

\begin{Shaded}
\begin{Highlighting}[]
\NormalTok{cell_sample3}
\end{Highlighting}
\end{Shaded}

\texttt{\#\#\ [1]\ \sout{\textcolor[rgb]{1.0,0.0,1.0}{68}}}\newline

\begin{Shaded}
\begin{Highlighting}[]
\NormalTok{cell_sample4}
\end{Highlighting}
\end{Shaded}

\texttt{\#\#\ [1]\ \textcolor[rgb]{0.0,1.0,0.0}{70}}\newline

\begin{Shaded}
\begin{Highlighting}[]
\NormalTok{vect_sample <-}\StringTok{ }\KeywordTok{color_vctr}\NormalTok{(cell_sample, cell_sample2, cell_sample3, cell_sample4)}
\NormalTok{vect_sample}
\end{Highlighting}
\end{Shaded}

\texttt{\#\#\ [1]\ \colorbox[rgb]{0.0,0.0,1.0}{\textcolor[rgb]{1.0,0.0,0.0}{24}}\ \colorbox[rgb]{1.0,1.0,0.0}{42}\ \sout{\textcolor[rgb]{1.0,0.0,1.0}{68}}\ \textcolor[rgb]{0.0,1.0,0.0}{70}}\newline

\begin{Shaded}
\begin{Highlighting}[]
\NormalTok{vect_sample2 <-}\StringTok{ }\NormalTok{vect_sample}
\NormalTok{vect_sample2[}\DecValTok{5}\NormalTok{] <-}\StringTok{ }\DecValTok{422}
\NormalTok{vect_sample2[}\DecValTok{20}\NormalTok{] <-}\StringTok{ }\KeywordTok{color_vctr}\NormalTok{(}\DecValTok{98119}\NormalTok{, }\DataTypeTok{text_color =} \StringTok{"yellow"}\NormalTok{, }\DataTypeTok{background =} \StringTok{"blue"}\NormalTok{, }\DataTypeTok{style =} \StringTok{"underline"}\NormalTok{)}
\NormalTok{vect_sample2[}\DecValTok{6}\OperatorTok{:}\DecValTok{7}\NormalTok{] <-}\StringTok{ }\KeywordTok{c}\NormalTok{(}\DecValTok{21}\NormalTok{,}\DecValTok{23}\NormalTok{)}
\NormalTok{vect_sample2[}\DecValTok{10}\OperatorTok{:}\DecValTok{12}\NormalTok{] <-}\StringTok{ }\KeywordTok{color_vctr}\NormalTok{(cell_sample, cell_sample2, cell_sample3)}

\NormalTok{vect_sample2}
\end{Highlighting}
\end{Shaded}

\texttt{\#\#\ \ [1]\ \colorbox[rgb]{0.0,0.0,1.0}{\textcolor[rgb]{1.0,0.0,0.0}{\ \ \ 24}}\ \colorbox[rgb]{1.0,1.0,0.0}{\ \ \ 42}\ \sout{\textcolor[rgb]{1.0,0.0,1.0}{\ \ \ 68}}\ \textcolor[rgb]{0.0,1.0,0.0}{\ \ \ 70}\ \ \ 422\ \ \ \ 21\ \ \ \ 23\ NA\ NA\ \colorbox[rgb]{0.0,0.0,1.0}{\textcolor[rgb]{1.0,0.0,0.0}{\ \ \ 24}}\ \colorbox[rgb]{1.0,1.0,0.0}{\ \ \ 42}\ \sout{\textcolor[rgb]{1.0,0.0,1.0}{\ \ \ 68}}}\newline
\texttt{\#\#\ [13]\ NA\ NA\ NA\ NA\ NA\ NA\ NA\ \colorbox[rgb]{0.0,0.0,1.0}{\underline{\textcolor[rgb]{1.0,1.0,0.0}{98119}}}}\newline

\begin{Shaded}
\begin{Highlighting}[]
\KeywordTok{data.frame}\NormalTok{(}\DataTypeTok{idx =} \DecValTok{1}\OperatorTok{:}\DecValTok{5}\NormalTok{, }\DataTypeTok{z =}\NormalTok{ vect_sample[}\DecValTok{1}\OperatorTok{:}\DecValTok{5}\NormalTok{])}
\end{Highlighting}
\end{Shaded}

\begin{longtable}[]{@{}ll@{}}
\toprule
idx & z\tabularnewline
\midrule
\endhead
1 &
\colorbox[rgb]{0.0,0.0,1.0}{\textcolor[rgb]{1.0,0.0,0.0}{24}}\tabularnewline
2 & \colorbox[rgb]{1.0,1.0,0.0}{42}\tabularnewline
3 & \sout{\textcolor[rgb]{1.0,0.0,1.0}{68}}\tabularnewline
4 & \textcolor[rgb]{0.0,1.0,0.0}{70}\tabularnewline
5 & NA\tabularnewline
\bottomrule
\end{longtable}

\begin{Shaded}
\begin{Highlighting}[]
\NormalTok{color_tibble <-}\StringTok{ }\KeywordTok{tibble}\NormalTok{(}\DataTypeTok{idx =} \DecValTok{1}\OperatorTok{:}\DecValTok{5}\NormalTok{, }\DataTypeTok{z =}\NormalTok{ vect_sample[}\DecValTok{1}\OperatorTok{:}\DecValTok{5}\NormalTok{]) }

\NormalTok{color_tibble}
\end{Highlighting}
\end{Shaded}

\begin{longtable}[]{@{}ll@{}}
\toprule
idx & z\tabularnewline
\midrule
\endhead
1 &
\colorbox[rgb]{0.0,0.0,1.0}{\textcolor[rgb]{1.0,0.0,0.0}{24}}\tabularnewline
2 & \colorbox[rgb]{1.0,1.0,0.0}{42}\tabularnewline
3 & \sout{\textcolor[rgb]{1.0,0.0,1.0}{68}}\tabularnewline
4 & \textcolor[rgb]{0.0,1.0,0.0}{70}\tabularnewline
5 & NA\tabularnewline
\bottomrule
\end{longtable}

\begin{Shaded}
\begin{Highlighting}[]
\NormalTok{color_tibble }\OperatorTok\StringTok{ }
\StringTok{  }\KeywordTok{kable}\NormalTok{(}\DataTypeTok{escape =} \OtherTok{FALSE}\NormalTok{)}
\end{Highlighting}
\end{Shaded}

\begin{longtable}[]{@{}rr@{}}
\toprule
idx & z\tabularnewline
\midrule
\endhead
1 &
\colorbox[rgb]{0.0,0.0,1.0}{\textcolor[rgb]{1.0,0.0,0.0}{24}}\tabularnewline
2 & \colorbox[rgb]{1.0,1.0,0.0}{42}\tabularnewline
3 & \sout{\textcolor[rgb]{1.0,0.0,1.0}{68}}\tabularnewline
4 & \textcolor[rgb]{0.0,1.0,0.0}{70}\tabularnewline
5 & NA\tabularnewline
\bottomrule
\end{longtable}

\hypertarget{use-cases}{%
\subsection{Use Cases}\label{use-cases}}

The ability to update coloring within the table allows for visualizing
the results before printing and rendering.

One use case could be trying to print out p-values and drawing attention
to the significant pvalues.

Normally, the course of action would be to manually add either the latex
or html required to tag the outputs. This requires both knowing how to
tag the significant pvalues with the correct latex/html code and also
hard codes those results into your code.

\begin{Shaded}
\begin{Highlighting}[]
\CommentTok{## Super Great analysis of mtcars!}

\NormalTok{lm_fit <-}\StringTok{ }\KeywordTok{lm}\NormalTok{(mpg }\OperatorTok{~}\StringTok{ }\NormalTok{., mtcars)}

\NormalTok{a_lm_fit <-}\StringTok{ }\KeywordTok{anova}\NormalTok{(lm_fit)}

\NormalTok{df_anova <-}\StringTok{ }\KeywordTok{data.frame}\NormalTok{(a_lm_fit)}

\CommentTok{# if the output is pdf}
\NormalTok{df_anova}\OperatorTok{$}\NormalTok{Pr..F. <-}\StringTok{ }\KeywordTok{ifelse}\NormalTok{(}
\NormalTok{  df_anova}\OperatorTok{$}\NormalTok{Pr..F. }\OperatorTok{<}\StringTok{ }\FloatTok{.05}\NormalTok{,}
  \KeywordTok{paste0}\NormalTok{(}\StringTok{"}\CharTok{\textbackslash{}\textbackslash{}}\StringTok{textcolor\{green\}\{"}\NormalTok{,df_anova}\OperatorTok{$}\NormalTok{Pr..F.,}\StringTok{"\}"}\NormalTok{),}
\NormalTok{  df_anova}\OperatorTok{$}\NormalTok{Pr..F.}
\NormalTok{)}

\KeywordTok{kable}\NormalTok{(df_anova)}
\end{Highlighting}
\end{Shaded}

\begin{longtable}[]{@{}lrrrrl@{}}
\toprule
& Df & Sum.Sq & Mean.Sq & F.value & Pr..F.\tabularnewline
\midrule
\endhead
cyl & 1 & 817.7129524 & 817.7129524 & 116.4245456 &
\textcolor{green}{5.03444973840481e-10}\tabularnewline
disp & 1 & 37.5939529 & 37.5939529 & 5.3525615 &
\textcolor{green}{0.0309108258078556}\tabularnewline
hp & 1 & 9.3709293 & 9.3709293 & 1.3342166 &
0.261031043915007\tabularnewline
drat & 1 & 16.4674349 & 16.4674349 & 2.3446047 &
0.140643762276576\tabularnewline
wt & 1 & 77.4757948 & 77.4757948 & 11.0308687 &
\textcolor{green}{0.00324449159445386}\tabularnewline
qsec & 1 & 3.9493082 & 3.9493082 & 0.5622956 &
0.461655702242183\tabularnewline
vs & 1 & 0.1297687 & 0.1297687 & 0.0184762 &
0.893173302477966\tabularnewline
am & 1 & 14.4742372 & 14.4742372 & 2.0608167 &
0.165857678951404\tabularnewline
gear & 1 & 0.9717105 & 0.9717105 & 0.1383504 &
0.71365333783354\tabularnewline
carb & 1 & 0.4066688 & 0.4066688 & 0.0579008 &
0.812178712952693\tabularnewline
Residuals & 21 & 147.4944300 & 7.0235443 & NA & NA\tabularnewline
\bottomrule
\end{longtable}

\{\{colortable\}\} can resolve this and make your code much easier to
understand, and you can add additional styling just as easily. There is
also the added benefit that even though we have styling on the cells,
the underlying object type still exists and can be modified and edited
as needed.

\begin{Shaded}
\begin{Highlighting}[]
\NormalTok{tbl_anova <-}\StringTok{ }\KeywordTok{data.frame}\NormalTok{(a_lm_fit)}

\NormalTok{tbl_anova}\OperatorTok{$}\NormalTok{Pr..F. <-}\StringTok{ }
\StringTok{  }\KeywordTok{set_styling}\NormalTok{(tbl_anova}\OperatorTok{$}\NormalTok{Pr..F. , tbl_anova}\OperatorTok{$}\NormalTok{Pr..F.  }\OperatorTok{<}\StringTok{ }\FloatTok{0.05}\NormalTok{, }\DataTypeTok{text_color =} \StringTok{"green"}\NormalTok{, }\DataTypeTok{style =} \StringTok{"underline"}\NormalTok{)}

\NormalTok{tbl_anova}
\end{Highlighting}
\end{Shaded}

\begin{longtable}[]{@{}llllll@{}}
\toprule
& Df & Sum.Sq & Mean.Sq & F.value & Pr..F.\tabularnewline
\midrule
\endhead
cyl & 1 & 817.7129524 & 817.7129524 & 116.42454564 &
\underline{\textcolor[rgb]{0.0,1.0,0.0}{5.034450e-10}}\tabularnewline
disp & 1 & 37.5939529 & 37.5939529 & 5.35256153 &
\underline{\textcolor[rgb]{0.0,1.0,0.0}{3.091083e-02}}\tabularnewline
hp & 1 & 9.3709293 & 9.3709293 & 1.33421658 &
2.610310e-01\tabularnewline
drat & 1 & 16.4674349 & 16.4674349 & 2.34460470 &
1.406438e-01\tabularnewline
wt & 1 & 77.4757948 & 77.4757948 & 11.03086869 &
\underline{\textcolor[rgb]{0.0,1.0,0.0}{3.244492e-03}}\tabularnewline
qsec & 1 & 3.9493082 & 3.9493082 & 0.56229561 &
4.616557e-01\tabularnewline
vs & 1 & 0.1297687 & 0.1297687 & 0.01847624 &
8.931733e-01\tabularnewline
am & 1 & 14.4742372 & 14.4742372 & 2.06081667 &
1.658577e-01\tabularnewline
gear & 1 & 0.9717105 & 0.9717105 & 0.13835045 &
7.136533e-01\tabularnewline
carb & 1 & 0.4066688 & 0.4066688 & 0.05790079 &
8.121787e-01\tabularnewline
Residuals & 21 & 147.4944300 & 7.0235443 & NA & NA\tabularnewline
\bottomrule
\end{longtable}

\end{document}
